\documentclass[notheorems]{beamer}
\usepackage{latexsym}
\usepackage{amsmath,amssymb}
\usepackage{color,xcolor}
\usepackage{graphicx}
\usepackage{algorithm}
\usepackage{amsthm}
\usepackage[UTF8]{ctex}
\usepackage{listings}

\usetheme{Szeged}
\usefonttheme[onlymath]{serif}
\usecolortheme{default}
\newtheorem{proposition}{命题}
\newtheorem{theorem}{定理}
\newtheorem{definition}{定义}
\newtheorem{example}{例}

\renewcommand\figurename{图}
\renewcommand\tablename{表}

\setbeamertemplate{theorems}[numbered]
\begin{document}

\title[字符串算法选讲]{字符串算法选讲}
\author[l0nl1f3]{l0nl1f3}
\institute[福州第三中学] {福州第三中学}
\date[June, 2017]{2017年6月}
\frame{\titlepage}

\begin{frame}
\frametitle{Tips}
\begin{itemize}[]  
\item 忘了Tips要写什么了
\end{itemize}
\end{frame}

\begin{frame}
\frametitle{KMP}
\begin{itemize}[]
\item 给定一个长度为$n$字符串$S$,给定一个长度为$m$字符串$T$,问$T$在$S$中出现了几次 
\end{itemize}
\end{frame}

\begin{frame}
\frametitle{KMP}
\begin{itemize}[]
\item 朴素匹配,枚举起点$l$,比较$s_{l\ldots l+m-1}$和$T$是否相等,遇到不等就移动$l$
\item 复杂度$O(nm)$
\pause
\item KMP(Knuth–Morris–Pratt algorithm),俗称看毛片算法
\pause
\item 定义一个串的border为满足$s_{1\ldots x} = s_{n-x+1\ldots n}$的前缀
\pause
\item KMP的next数组存储的是每个前缀的最大border的长度
\end{itemize}
\end{frame}

\end{document}