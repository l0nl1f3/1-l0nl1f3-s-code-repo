\documentclass[notheorems]{beamer}
\usepackage{latexsym}
\usepackage{amsmath,amssymb}
\usepackage{color,xcolor}
\usepackage{graphicx}
\usepackage{algorithm}
\usepackage{amsthm}
\usepackage[UTF8]{ctex}
\usepackage{listings}

\usetheme{Szeged}
\usefonttheme[onlymath]{serif}
\usecolortheme{default}
\newtheorem{proposition}{命题}
\newtheorem{theorem}{定理}
\newtheorem{definition}{定义}
\newtheorem{example}{例}

\renewcommand\figurename{图}
\renewcommand\tablename{表}

\setbeamertemplate{theorems}[numbered]
\begin{document}

\title[字符串算法选讲]{字符串算法选讲}
\author[l0nl1f3]{l0nl1f3}
\institute[福州第三中学] {福州第三中学}
\date[June, 2017]{2017年6月}
\frame{\titlepage}

\begin{frame}
\frametitle{Tips}
\begin{itemize}[]  
\item 忘了Tips要写什么了
\end{itemize}
\end{frame}

\begin{frame}
\frametitle{KMP}
\begin{itemize}[]
\item 给定一个长度为$n$字符串$S$,给定一个长度为$m$字符串$T$,问$T$在$S$中出现了几次 
\pause
\item $n\leq m\leq 5*10^6$
\end{itemize}
\end{frame}

\begin{frame}
\frametitle{KMP}
\begin{itemize}[]
\item 朴素匹配,枚举起点$l$,比较$s_{l\ldots l+m-1}$和$T$是否相等,遇到不等(失配)就移动$l$
\item 复杂度$O(nm)$
\pause
\item KMP(Knuth–Morris–Pratt algorithm),俗称看毛片算法
\pause
\item 定义一个串的border为满足$s_{1\ldots x} = s_{n-x+1\ldots n}$的前缀
\pause
\item KMP的next数组存储的是$T$串每个前缀的最大border的长度
\end{itemize}
\end{frame}

\begin{frame}
\frametitle{KMP}
\begin{itemize}[]
\item 失配时按border跳
\pause
\item 构建next数组$O(n)$,匹配总复杂度$O(n)$
\end{itemize}
\end{frame}

\begin{frame}
\frametitle{KMP}
\begin{itemize}[]
\item 给定一个长度为$n$字符串$S$,给定一个长度为$m$字符串$T$,问$T$在$S$中出现了几次 
\pause
\item $n\leq m\leq 5*10^6$
\end{itemize}
\end{frame}

\begin{frame}
\frametitle{Easy Period Problem}
\begin{itemize}[]
\item 对于一个字符串$T$,如果存在字符串$A$,使得$A+A+\ldots (x个A)= T$,其中加法为顺次连接
\item 则称$T=A^x$,求满足条件的最短$A$的长度
\pause
\item 设$T=n-next_n$,若$n \bmod T = 0$,则答案为$T$
\item 正确性如何?
\end{itemize}
\end{frame}

\begin{frame}
\frametitle{Trie}
\begin{itemize}[]
\item 给定$n$个串,第$i$个串长为$l_i$,$Q$次询问$(x,y)$的最长公共前缀
\item $\sum l_i \leq 5*10^6,Q\leq 10^6,n\leq 10^5$
\pause
\item 建立一棵26叉前缀树(trie).每条边上有一个字母
\item 根到某个点的路径组成一个前缀
\item 每次插入一个字符串时,就在trie树上把对应路径“填满”,并记录末字符所在的节点
\pause
\item 将公共前缀查询转化为LCA查询
\item $O(\sum_{l_i}\log \sum{l_i} + Q)$
\end{itemize}
\end{frame}

\begin{frame}
\frametitle{Trie + KMP}
\begin{itemize}[]
\item 给定$n$个串,第$i$个串长为$l_i$
\item 再给定一个长度为$m$的文本串,问有多少个串在文本串中出现过
\item $\sum l_i \leq 5*10^5,m\leq 10^6,n\leq 10^4$
\pause
\item 大力KMP,复杂度$O(nm)$
\end{itemize}
\end{frame}

\begin{frame}
\frametitle{Trie + KMP}
\begin{itemize}[]
\item Trie + KMP = Aho-Corasick Automation
\pause
\item 对于Trie上的每一个前缀,我们任然可以建立一个像kmp一样的next数组
\item 对于Trie进行bfs,考虑将要遍历$i$节点的$c$孩子
\pause
\item 若$c$孩子不存在,则将孩子指针指向$next_i$的$c$孩子
\item 否则将该孩子的$next$指向$next_i$节点的$c$孩子
\pause
\item 记录每个点的$cnt$值为整串结尾在该点串数
\end{itemize}
\end{frame}

\begin{frame}
\frametitle{Trie + KMP}
\begin{itemize}[]
\item 那么我们对文本串进行一个Trie上的kmp即可
\item $O(\sum l_i + m)$
\item HDU2222
\end{itemize}
\end{frame}

\begin{frame}
\frametitle{Trie + KMP - Trie}
\begin{itemize}[]
\item 给出n个字符串,询问每个字符串在所有字符串中出现的次数之和
\item $n\leq 10^5,\sum l_i \leq 10^6$
\pause
\item 暴力AC自动机的复杂度???
\pause
\item $O((\sum l_i)^2)$
\pause
\item 我们建立一棵fail树,我们把每个节点$x$向$next_x$连边
\pause
\item 每个点都是一个字符串的前缀,而且每个字符串的每个前缀在这棵树上都对应着一个点。
\item 其次,由于fail指针,每个点父节点的字符串都是这个点字符串最长的存在的后缀
\end{itemize}
\end{frame}


\begin{frame}
\frametitle{Trie + KMP - Trie}
\begin{itemize}[]
\item 插入字符串时记录每个点被经过的次数
\item 最后在fail树上答案体现为子树和
\item 复杂度$O(n+\sum l_i)$,BZOJ3172
\end{itemize}
\end{frame}

\begin{frame}
\frametitle{Trie + KMP - Trie}
\begin{itemize}[]
\item 插入字符串时记录每个点被经过的次数
\item 最后在fail树上答案体现为子树和
\item 复杂度$O(n+\sum l_i)$,BZOJ3172
\end{itemize}
\end{frame}



\end{document}